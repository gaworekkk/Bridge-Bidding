\documentclass{article}
\usepackage{graphicx}
\usepackage[polish]{babel}
\usepackage[utf8]{inputenc}
\usepackage[T1]{fontenc}
\usepackage{enumitem}
\usepackage{hyperref}

\title{Specyfikacja techniczna systemu do licytacji brydżowej}
\author{Wojciech Romanowski}
\date{Czerwiec 2025}

\begin{document}

\maketitle

\tableofcontents

\section{Wprowadzenie}
Dokument opisuje specyfikację techniczną systemu do licytacji brydżowej, który umożliwia rozgrywanie wirtualnych partii brydża z pełnym procesem licytacji i obliczania wyników.

\section{Opis klas}

\subsection{Player}
Klasa reprezentująca gracza.

\begin{itemize}
    \item Pola:
    \begin{itemize}
        \item \texttt{String nick} - nazwa gracza
    \end{itemize}
    
    \item Metody:
    \begin{itemize}
        \item \texttt{Player(String nick)} - konstruktor tworzący gracza o podanym pseudonimie
    \end{itemize}
\end{itemize}

\subsection{Team}
Klasa reprezentująca drużynę (parę graczy).

\begin{itemize}
    \item Pola:
    \begin{itemize}
        \item \texttt{String name} - nazwa drużyny
        \item \texttt{Player player1, player2} - gracze w drużynie
        \item \texttt{List<Integer> scores} - lista wyników z poszczególnych rund
        \item \texttt{int totalScore} - suma punktów drużyny
    \end{itemize}
    
    \item Metody:
    \begin{itemize}
        \item \texttt{Team(Player p1, Player p2, String name)} - konstruktor tworzący drużynę
        \item \texttt{void addScore(int score)} - dodaje wynik do historii drużyny
        \item \texttt{int average\_score()} - oblicza średni wynik drużyny
        \item \texttt{void displayStats()} - wyświetla statystyki drużyny
    \end{itemize}
\end{itemize}

\subsection{Round}
Klasa reprezentująca rundę gry.

\begin{itemize}
    \item Pola:
    \begin{itemize}
        \item \texttt{List<Bidding> biddings} - lista odzywiek w licytacji
        \item \texttt{Contract contract} - kontrakt ustalony w rundzie
        \item \texttt{int nsScore, ewScore} - wyniki drużyn NS i EW
        \item \texttt{int roundNumber} - numer rundy
    \end{itemize}
    
    \item Metody:
    \begin{itemize}
        \item \texttt{Round(int roundNumber)} - konstruktor tworzący rundę
        \item \texttt{void addBidding(Bidding bid)} - dodaje odzywkę do licytacji
    \end{itemize}
\end{itemize}

\subsection{Game}
Klasa zarządzająca całą grą.

\begin{itemize}
    \item Pola:
    \begin{itemize}
        \item \texttt{Team teamNS, teamEW} - drużyny North-South i East-West
        \item \texttt{int scoreNS, scoreEW} - aktualne wyniki drużyn
        \item \texttt{List<List<Bidding>> biddingSequence} - sekwencje licytacji
        \item \texttt{List<Round> rounds} - lista rund
        \item \texttt{Table table} - układ graczy przy stole
        \item \texttt{int currentRound, maxRounds} - aktualna i maksymalna liczba rund
        \item \texttt{boolean gameFinished} - czy gra zakończona
        \item \texttt{boolean nsVulnerable, ewVulnerable} - status "vulnerable" drużyn
        \item \texttt{LocalDateTime startTime, endTime} - czas rozpoczęcia i zakończenia
    \end{itemize}
    
    \item Metody:
    \begin{itemize}
        \item \texttt{Game(Team ns, Team ew, Table table, int maxRounds)} - konstruktor
        \item \texttt{void add\_bidding(int round, Bidding bid)} - dodaje odzywkę
        \item \texttt{void displayBidding()} - wyświetla historię licytacji
        \item \texttt{boolean nextRound()} - rozpoczyna nową rundę
        \item \texttt{void updateVulnerability()} - aktualizuje status vulnerable
        \item \texttt{void displayScoreHistory()} - pokazuje historię wyników
        \item \texttt{void displayRoundSummary()} - podsumowuje rundę
        \item \texttt{Round getCurrentRound()} - zwraca aktualną rundę
        \item \texttt{void startNewRound()} - inicjalizuje nową rundę
        \item \texttt{String getGameInfo()} - zwraca podstawowe informacje o grze
        \item \texttt{String getFullGameData()} - zwraca pełne dane gry do zapisu
    \end{itemize}
\end{itemize}

\subsection{Scoreboard}
Klasa reprezentująca tablicę wyników.

\begin{itemize}
    \item Pola:
    \begin{itemize}
        \item \texttt{int round} - numer rundy
    \end{itemize}
    
    \item Metody:
    \begin{itemize}
        \item \texttt{void displayScores(Team ns, Team ew, int scoreNS, int scoreEW)} - wyświetla wyniki
    \end{itemize}
\end{itemize}

\subsection{Table}
Klasa reprezentująca stół do gry.

\begin{itemize}
    \item Pola:
    \begin{itemize}
        \item \texttt{Player north, east, south, west} - gracze na pozycjach
    \end{itemize}
    
    \item Metody:
    \begin{itemize}
        \item \texttt{void assignPositions(Player n, Player e, Player s, Player w)} - przypisuje graczy
        \item \texttt{Team getNSTeam()} - zwraca drużynę NS
        \item \texttt{Team getEWTeam()} - zwraca drużynę EW
    \end{itemize}
\end{itemize}

\subsection{Bidding}
Klasa reprezentująca odzywkę w licytacji.

\begin{itemize}
    \item Pola:
    \begin{itemize}
        \item \texttt{Player player} - gracz składający odzywkę
        \item \texttt{String bid} - treść odzywki
    \end{itemize}
    
    \item Metody:
    \begin{itemize}
        \item \texttt{Bidding(Player player, String bid)} - konstruktor
        \item \texttt{String toString()} - reprezentacja tekstowa odzywki
    \end{itemize}
\end{itemize}

\subsection{Contract}
Klasa reprezentująca kontrakt brydżowy.

\begin{itemize}
    \item Pola:
    \begin{itemize}
        \item \texttt{Team declaringTeam, undeclaringTeam} - drużyny
        \item \texttt{int level} - poziom kontraktu
        \item \texttt{String suit} - kolor
        \item \texttt{int result} - wynik (nadróbki/niedóróbki)
        \item \texttt{boolean doubled, redoubled} - czy kontra/rekontra
    \end{itemize}
    
    \item Metody:
    \begin{itemize}
        \item \texttt{Contract(Team d, Team u, int level, String suit, int result, boolean doubled, boolean redoubled)} - konstruktor
        \item \texttt{int calculateScore(boolean isDeclarerVulnerable)} - oblicza wynik
        \item \texttt{int calculateContractPoints(int level, String suit)} - oblicza punkty za kontrakt
        \item \texttt{int colorValue(String color)} - wartość koloru
        \item \texttt{int calculatePenalty(int undertricks, boolean doubled, boolean redoubled, boolean vulnerable)} - oblicza kary
    \end{itemize}
\end{itemize}

\subsection{History}
Klasa zarządzająca historią gier.

\begin{itemize}
    \item Pola:
    \begin{itemize}
        \item \texttt{List<Game> gamesHistory} - lista rozegranych gier
        \item \texttt{Scoreboard scoreboard} - tablica wyników
        \item \texttt{private static final String HISTORY\_FILE} - plik historii
    \end{itemize}
    
    \item Metody:
    \begin{itemize}
        \item \texttt{History()} - konstruktor
        \item \texttt{void add\_game(Game game)} - dodaje grę do historii
        \item \texttt{void displayGamesList()} - wyświetla listę gier
        \item \texttt{void displayGameDetails(int gameIndex)} - pokazuje szczegóły gry
        \item \texttt{private void saveHistoryToFile()} - zapisuje historię do pliku
        \item \texttt{private void loadHistoryFromFile()} - wczytuje historię z pliku
    \end{itemize}
\end{itemize}

\subsection{GameManager}
Główna klasa zarządzająca rozgrywką.

\begin{itemize}
    \item Pola:
    \begin{itemize}
        \item \texttt{private List<Player> players} - lista graczy
        \item \texttt{History history} - historia gier
        \item \texttt{private Scanner scanner} - do odczytu wejścia
    \end{itemize}
    
    \item Metody:
    \begin{itemize}
        \item \texttt{GameManager()} - konstruktor
        \item \texttt{void start()} - główna pętla programu
        \item \texttt{private void playNewGame()} - rozpoczyna nową grę
        \item \texttt{private void setupPlayers()} - konfiguruje graczy
        \item \texttt{Game setupGame()} - przygotowuje grę
        \item \texttt{void playBidding(Game game)} - przeprowadza licytację
        \item \texttt{private void createContract(Game game, String lastBid, boolean doubled, boolean redoubled)} - tworzy kontrakt
        \item \texttt{boolean isValidBid(String bid)} - sprawdza poprawność odzywki
        \item \texttt{boolean isHigherBid(String newBid, String lastBid)} - porównuje odzywki
        \item \texttt{int parseResult(String resultStr)} - parsuje wynik
        \item \texttt{void browseHistory()} - przegląda historię
        \item \texttt{void displayGameSummary(Game game)} - podsumowuje grę
        \item \texttt{void info()} - wyświetla informacje pomocnicze
    \end{itemize}
\end{itemize}

\subsection{BridgeBidding}
Klasa główna programu.

\begin{itemize}
    \item Metody:
    \begin{itemize}
        \item \texttt{void main(String[] args)} - punkt wejścia programu
    \end{itemize}
\end{itemize}

\section{Opis funkcjonalności}
Program implementuje następujące funkcjonalności:
\begin{itemize}
    \item Tworzenie graczy i drużyn
    \item Przeprowadzanie licytacji zgodnie z zasadami brydża
    \item Obliczanie wyników kontraktów z uwzględnieniem kontr i rekontr
    \item Śledzenie statusu vulnerable drużyn
    \item Zapisywanie i wczytywanie historii rozgrywek
    \item Generowanie statystyk i podsumowań
\end{itemize}

\end{document}
